\chapter{Conclusion and prospect}\label{sec-conclusion}

\section{Statistics result of coherent eddies in the Argentine Basin}


There is an upward trend of eddies in the research domain. Lots of eddies have been observed in the southwest part of the AB and the center part of the AB is the 'Grave' of the Vortex. When we extend the coherence time, the percentage of anticyclonic eddies starts to become larger.

Eulerian eddy's radius is almost twice as large as a Lagrangian coherent eddy.

\section{Three-dimensional structure of coherent eddies}

There are three types of oceanic eddy and lens-shaped eddy is the most common type. Oceanic eddies can strongly affect the thermohaline properties of seawater. Cyclonic eddy is deeper than anticyclonic eddy.

\section{Future work plan}



Although some progress has been made in the work of this paper, there are still many issues that need to be studied and discussed further.

My future research plan mainly focuses on the dynamic mechanism that is related to the formation and temporal and spatial variation of eddies in the study region and how oceanic eddies modify water mass properties and have a far-reaching impact on the regional and global climate systems. The future work plans are as follows:

\begin{itemize}
  \item [1)] 
  Model data may have large differences in matching with satellite data, and there are limitations in constructing 3D eddies. In the future, we can have a more in-depth understanding of the 3D structure of eddies and the temperature and salt fields from continuous observations of field data such as Argo buoy profiles.
  \item [2)]
  The relationship between mesoscale eddies and the background flow field, and the characteristics of eddies under different climatic background conditions need further study.
  \item [3)]
  The transport capacity of vortex-induced heat, energy, and water masses on long-time scales needs to be studied in depth.
  \item[4)]
  Other relevant and interdisciplinary studies such as interactions and feedback processes between oceanic eddies and the atmosphere, effects of eddies on the marine ecosystem, application of extraction of the skeleton in the geophysical flow field, etc.
\end{itemize}




\newpage

