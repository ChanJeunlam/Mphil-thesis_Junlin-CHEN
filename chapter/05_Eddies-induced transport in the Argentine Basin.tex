\chapter{Eddies-induced transport in the Argentine Basin}

In this chapter, we present statistical analysis

\section{Introduction}

From the analysis result as shown in Chapter \ref{Eddies features analysis} and eddy vertical type in Chapter \ref{Three-dimensional eddies structures}, we gain insight into seasonal and annual change of surface eddy properties and eddy structure. However, more information about eddies' transport ability is needed to know because understanding water mass exchange of several important tracers (e.g., heat and salt) is crucial to eddy mixing and eddy parameterization \cite{guan2022seasonal}.

Volume, heat and salt transport carried by oceanic eddies are important source of water exchange especially in the Southern Ocean and Argentine basin. However, it is still under debate how much eddy-induced transport would contribute to the overall heat and salt transport and how it would change the local climate since most of the low-resolution climate models could not resolve eddies. In previous studies based on Eulerian methods, it is declared that eddy-induced zonal transport is comparable with the transports of wind-driven gyres and thermohaline circulation \cite{zhang2014oceanic} and eddy-induced meridional transport account for $20\%-30\%$ of the overall volume transport\cite{dong2014global}. However, studies based on Lagrangian approaches come to the conclusion that coherent eddy-induced material transport is smaller than Eulerian-based estimation by two magnitudes and its contribution to thermodynamics budget is less significant \cite{wang2015coherent}. 

Sea level anomaly from satellite can provides us with the mesoscale dynamical properties of mesoscale eddies; however, it could not give us subsurface information of eddies structures, let alone heat and salt transport induced by eddies. What is more, eddies-induced vertical movement of seawater can greatly deform the thermohaline structure and cause temperature and salinity anomalies inside eddies. Thus, it is of great importance to understand how eddies disturb the background salinity and temperature field and transport heat, salt and energy across the basin.

More importantly, eddies' crucial role in the large scale circulation and its distinct hydrographic properties compared with the surrounding waters influence a lot of oceanic processes such as energy cascade from mean kinetic energy (MKE) to eddy kinetic energy (EKE), upwelling or downwelling, etc.

According to the previous studies, eddy-induced mixing may contribute as much as the mean flow advection to the large-scale water mass subduction in Brazil–Malvinas Confluence region \cite{marshall2006estimates}.

\section{Estimate of eddy-induced heat and salinity transport}

There are currently three main Eulerian methods to calculate eddy-induced heat (salt) transport. The first one is direct calculation of the product of temperature (salt) anomaly and velocity anomaly $\overline{T^{\prime} V^{\prime}}\left(\overline{S^{\prime} V^{\prime}}\right)$ \cite{volkov2008eddy}. This approach requires long-term climate-state data to extract deviation from the mean field ($T = \overline{T}+ T^{\prime}$). Assuming a linear relationship between the temperature (salt) perturbation flux and the mean state field gradient, similar to the turbulent Reynolds stress relationship, thus avoiding deviated field calculations, we get the second estimation method: $\overline{T^{\prime} V^{\prime}} =k \frac{\overline{\partial T}}{\partial y}$ \cite{stammer1998eddy,chen2012eddy}. However, this estimation relies on the selection of an empirical eddy diffusion coefficient $k$.

Above methods still have their limitations because the Eulerian framework do not consider the the kinematic characteristics of the vortex as a whole. In fact, because of the highly nonlinear nature of the vortex, coherent eddy is different from the general perturbation. If we trace the vortex, the temperature and salt anomalies caused by the vortex will maintain a certain stability inside the vortex during its life cycle along the vortex trajectory. In other words, the temperature and salt anomalies can be carried and advected by vortex. This way of following the vortex trajectory can only be clearly and precisely described in the Lagrangian framework.


\section{Eddy-induced heat transport}

\section{Eddy-induced salt transport}

eddies fresher isopycnal layer salinity anomalies

all the screened eddies

\section{Eddies-eddies interaction and eddies-mean flow interaction}

From eddies footprint, we could conclude that eddies absorb into Brazil Current.

\newpage
